\documentclass[letter]{book}

%%%%%% Import Package %%%%%%
\usepackage{graphicx}
\usepackage[unicode]{hyperref}
\usepackage{cite}
\usepackage{indentfirst}
\usepackage{multirow}
\usepackage{indentfirst}
\usepackage{titlesec}
\usepackage{xcolor}
\usepackage{listings}
\usepackage{fontspec,xunicode,xltxtra}
\usepackage{xeCJK}
\usepackage{hyperref}
\usepackage{enumerate}
\usepackage{epigraph}
\usepackage{amsmath}
\usepackage[xindy]{glossaries}
\usepackage{fancyhdr}
\usepackage{amsmath}
%\usepackage{TikZ}
\usepackage{ifthen}
\usepackage{longtable}

%%%% 下面的命令设置行间距与段落间距 %%%%
\linespread{1.4}
% \setlength{\parskip}{1ex}
\setlength{\parskip}{0.5\baselineskip}

%Set Code Format%
\lstloadlanguages{C, csh, make,python,Java}
\lstset{	  
	alsolanguage= XML,  
	tabsize=4, %  
	frame=shadowbox, %把代码用带有阴影的框圈起来  
	commentstyle=\color{red!50!green!50!blue!50},%浅灰色的注释  
	rulesepcolor=\color{red!20!green!20!blue!20},%代码块边框为淡青色  
	keywordstyle=\color{blue!90}\bfseries, %代码关键字的颜色为蓝色,粗体  
	showstringspaces=false,%不显示代码字符串中间的空格标记  
	stringstyle=\ttfamily, % 代码字符串的特殊格式  
	keepspaces=true, %  
	breakindent=22pt, %  
	numbers=left,%左侧显示行号 往左靠,还可以为right,或none,即不加行号  
	stepnumber=1,%若设置为2,则显示行号为1,3,5,即stepnumber为公差,默认stepnumber=1  
	%numberstyle=\tiny, %行号字体用小号  
	numberstyle={\color[RGB]{0,192,192}\tiny} ,%设置行号的大小,大小有tiny,scriptsize,footnotesize,small,normalsize,large等  
	numbersep=8pt,  %设置行号与代码的距离,默认是5pt  
	basicstyle=\footnotesize, % 这句设置代码的大小  
	showspaces=false, %  
	flexiblecolumns=true, %  
	breaklines=true, %对过长的代码自动换行  
	breakautoindent=true,%  
	breakindent=4em, %  	   
	aboveskip=1em, %代码块边框  
	tabsize=4,  
	showstringspaces=false, %不显示字符串中的空格  
	backgroundcolor=\color[RGB]{245,245,244},   %代码背景色  
	%backgroundcolor=\color[rgb]{0.91,0.91,0.91}    %添加背景色  
	escapeinside=``,  %在``里显示中文  
	%% added by http://bbs.ctex.org/viewthread.php?tid=53451  
	fontadjust,  
	captionpos=t,  
	framextopmargin=2pt,framexbottommargin=2pt,abovecaptionskip=-3pt,belowcaptionskip=3pt,  
	xleftmargin=4em,xrightmargin=4em, % 设定listing左右的空白  
	texcl=true
}



\begin{document}

\section*{Problem 1}

\subsection*{Algorithm}
dsasadsadsadsad年
\newpage
\section*{Groovy}


\newpage
\section*{Gradle}

\subsection{执行流程}

There is a one-to-one relationship between a Project and a "build.gradle" file. During build initialisation, Gradle assembles a Project object for each project which is to participate in the build, as follows:

Create a Settings instance for the build.
Evaluate the "settings.gradle" script, if present, against the Settings object to configure it.
Use the configured Settings object to create the hierarchy of Project instances.
Finally, evaluate each Project by executing its "build.gradle" file, if present, against the project. The projects are evaluated in breadth-wise order, such that a project is evaluated before its child projects. This order can be overridden by calling evaluationDependsOnChildren() or by adding an explicit evaluation dependency using evaluationDependsOn(String).

\subsection{repositories}

在Gradle构建标本build.gradle里,经常会看到如下脚本:

\begin{lstlisting}[language=Java]
repositories {
	maven {
		url 'http://www.eveoh.nl/files/maven2'
	}
	maven {
		url 'http://repox.gtan.com:8078'
	}
	mavenCentral()
	jcenter()
	maven { url 'http://repo.spring.io/plugins-release' }
}
\end{lstlisting}

总的来说,只有两个标准的Android library文件服务器:Jcenter 和 Maven Central。起初,Android Studio 选择Maven Central作为默认仓库。如果你使用老版本的Android Studio创建一个新项目,mavenCentral()会自动的定义在build.gradle中。但是Maven Central的最大问题是对开发者不够友好。上传library异常困难。上传上去的开发者都是某种程度的极客。同时还因为诸如安全方面的其他原因,Android Studio团队决定把默认的仓库替换成jcenter。正如你看到的,一旦使用最新版本的Android Studio创建一个项目,jcenter()自动被定义,而不是mavenCentral()。mavenCentral()表示依赖是从Central Maven 2 仓库中获取的,库的地址是\url{https://repo1.maven.org/maven2}。jcenter表示依赖是从Bintary’s JCenter Maven 仓库中获取的,仓库的地址是\url{https://jcenter.bintray.com},bintray是一家提供全球企业软件开发包托管的商业公司。

\subsection{属性(Properties)}

\subsubsection{Extra Properties}

extra属性一般用于定义常量,All extra properties\footnote{\url{https://docs.gradle.org/current/javadoc/org/gradle/api/Project.html\#extraproperties}} must be defined through the "ext" namespace. Once an extra property has been defined, it is available directly on the owning object (in the below case the Project, Task, and sub-projects respectively) and can be read and updated. Only the initial declaration that needs to be done via the namespace.

\begin{lstlisting}[language=Java]
buildscript {
	ext {
		springBootVersion = '1.4.5.RELEASE'
		jacksonVersion = '2.8.7'
		springfoxVersion = '2.6.1'
		poiVersion = "3.14"
		aspectjVersion = '1.7.4'
	}
}
\end{lstlisting}

Reading extra properties is done through the "ext" or through the owning object.
ext.isSnapshot = version.endsWith("-SNAPSHOT")
if (isSnapshot) {
	// do snapshot stuff
}


\section{LaTex}

\subsection{字体}

Computer Modern是自由软件TeX的默认字体,为美国计算机科学家高德纳(Donald Knuth)使用METAFONT软件创造。但是此字体不是很漂亮,所以考虑换一个字体 。


\end{document}
\theend